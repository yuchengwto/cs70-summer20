\documentclass{article}
\usepackage{mathtools}
\usepackage{amsfonts}
\usepackage{amssymb}
\usepackage{enumerate}
\usepackage{amsthm}
\usepackage{amsmath}
\usepackage{listings}
\newcommand{\powerset}[1]{\mathbb{P}(#1)}

\begin{document}
\part*{Countability Practice}
    \begin{enumerate}[(a).]
        \item Yes, they have the same cardinality.
        \[f(x) = \frac{1}{x} - 1\]
        Suppose f(x) = f(y)\\
        \begin{align*}
            &\frac{1}{x} - 1 = \frac{1}{y} - 1\\
            &x = y\\
        \end{align*}
        Hence f is one to one.\\
        \\
        Take any $y \in (0, \infty)$, let $x \in (0, 1), x = 1/(1+y)$. Then,
        \[f(x) = \frac{1}{x} - 1 \]
        \[f(x) = \frac{1}{\frac{1}{1+y}} - 1\]
        \[f(x) =  y\]
        So f maps x to y, hence onto.

        \item Yes, it is countable. We can enumerate the strings in 
        such a way that each string appears exactly once in the list. 
        Firstly, list all strings of length 1 in lexicographic order, then 
        all strings of length 2, and then length 3, and so on. Since each step, 
        there are only finite number of strings of a particular length, any string 
        of finite length appears in the list, and it is clear that each string 
        appears exactly once in this list.

        
        \item The answer do not change, the strings are still countable. 
        We can encode the strings into ternary strings. Suppose we have a string: S = $a_5a_2a_7a_4a6$. 
        Corresponding to each of the characters in this string, we can write its index as a binary string: 
        (101, 10, 111, 100, 110). Moreover, we can construct a ternary string where "2" is inserted as a separator 
        between each binary string. Thus we can map the string S to a ternary string: 
        101210211121002110. This map is injective, since the original string S can be 
        uniquely recovered from this ternary string. Thus we have an injective map to $\{0, 1, 2\}^*$. 
        Since the $\{0, 1, 2\}^*$ is countable, hence the set of all strings with finite 
        length is also countable.


    \end{enumerate}    

\part*{Fixed Points}
We can prove this by reducing from the Halting Problem. 
Suppose we had some function FixedPoint(F) 
that solved the fixed-point problem. 
We can define TestHalt(F,x) as follows:
\begin{lstlisting}
    def TestHalt(F, x):
        def F_prime(y):
            F(x)
            return y
        return FixedPoint(F_prime)
\end{lstlisting}

If $F(x)$ halt, then $F_prime(y)$ return y, and $FixedPoint(F_prime)$ 
will return true. If $F(x)$ does not halt, then $F_prime(y)$ does not 
return y, hence $FixedPoint(F_prime)$ return false. Thus TestHalt 
works! This is contradiction, we conclude.


\part*{The Complexity Hierarchy}
\begin{enumerate}[(a).]
    \item 
    \emph{enumerator:}
    \begin{lstlisting}
        for i = 1 to \infty:
            foreach program M where len(M) < l:
                foreach input x where len(x) < l:
                    simulate M(x) for at most l steps
                        if M(x) has halted, then print out (M,x)
    \end{lstlisting}
    
    \item 
    \emph{recognizer:}
    \begin{lstlisting}
        run M(x)
        return Yes
    \end{lstlisting}

    \item 
    The desired output for input(M,x) is No if M halts on x, 
    and Yes if M does not halt on x.

    \item 
    Suppose there exists a recognizer Q for the Halting Problem, 
    also a recognizer Q' for the complement. We can define TestHalt(P,x): 
    \begin{lstlisting}
        Alternate between simulating one step of Q(P,x) and Q'(P, x)
        if Q(P, x) returns Yes, return Yes
        if Q'(P, x) returns Yes, return No
    \end{lstlisting}
    We can guarantee that one of Q and Q' will evetually return Yes. 
    Hence the TestHalt will eventually print out Yes if P(x) halts, and No otherwise. 
    However, the TestHalt is not computable, so we have a contradiction. 
    Therefore the Halting Problem cannot be in coRE.
\end{enumerate}

\part*{Build Up Error}
    In the \emph{Inductive Step}, the (n+1)-vertex graph can not only be obtained from 
    a n-vertex graph with minimum 1 degree, but also some graph with 0 degree.
    
\part*{Connectivity}

\begin{enumerate}[(a).]
    \item For any two non-adjacent vertices u and v in a graph with n vertices. 
    When deg u + deg v >= n - 1, assume there is no vertex w connected to both u and v, 
    then there only leaves n - 2 vertices for u and v to connect, hence deg u + deg v <= n - 2.
    We get a contradiction, so there must be at least one vertex w connected to u and v simutaneously, 
    hence the graph is connected.

    \item There are only two non-adjacent vertices u and v, 
    which satisfies the condition deg u + deg v = 0 = 2 - 2 >= n - 2. 
    But the graph is not connected.

    \item For any two non-adjacent vertices u and v, deg u + deg v >= 
    n/2 + n/2 = n > n - 1, by part(a), hence the graph is connected.

    \item There are exactly two vertices with odd degree. 
    Assume they are not connected, then they are in two components, 
    and each component has only one vertex with odd degree and therefore 
    an odd sum of degree. But a connected component can only has an even sum of degree, which is contradict. 
    Hence, we conclude.
\end{enumerate}

\part*{Triangular Faces}
We can prove by contradiction. Assume there exists one face which is incident on more than three edges. 
Choose an arbitrary vertex on the face and name it $v_0$, then with clockwise from $v_0$, 
there is $v_1$, $v_2$, $v_3$... Since the face has at least 4 edges, so $v_0$ and $v_2$ do not have an edge. 
If we add an edge between $v_0$ and $v_2$, the edge does not cross any other edges and the graph is still planar. 
Thus the number of edges is $e + 1 = 3v - 5$, which is contradicted to the theorem that a planar graph can have at most 3v - 6 edges.

\part*{Binary Tree}
\begin{enumerate}[(a).]
    \item Suppose $u \in L$. Before removing r, u had degree 1 or 3. 
    If u had degree 1, then after removing r, u is a single vertex, and so a binary tree of height 0. 
    If u had degree 3, then after removing r, u has degree 2, and all other vertices in L have degree 1 or 3. 
    Thus, L is a new binary tree with root u. 
    \\
    Formally, we define $\Omega(L)$ and $\Omega(R)$ to be the set of leaves in L and R respectively, 
    and d(r, l) as the length of the path from r to some leaf l, then we have 
    \[h(T) = \max(1 + \mathop{\max(d(u,l))}\limits_{l \in \Omega(L)}, 1 + \mathop{\max(d(v,l))}\limits_{l \in \Omega(R)} \]
    \[ = 1 + \max(h(L),h(R)) \]

    \item Induction on height h:\\
    \emph{Base Case: }h = 0, there is $2^{0+1} - 1 = 1$ vertex.\\
    \emph{Inductive Hypothesis: }Assume $\forall k < h, v \le 2^{k+1} - 1$.\\
    \emph{Inductive Step: }By part a, since $h(T) = \max(h(L),h(R)) + 1$, 
    so h(L), h(R) < h, and we can apply the Inductive Hypothesis to L and R. 
    Thus, the number of vertices in T is at most $2^{h(L)+1} - 1 + 2^{h(R)+1} - 1 + 1 = 
    2^{k+1} + 2^{h(L)+1} + 2^{h(R)+1} - 1 \le 2*2^{h(T)} - 1 = 2^{h(T)+1} - 1$

    \item Induction on the number of leaf l:\\
    \emph{Base Case: }l = 1, there is $2*1 - 1 = 1$ vertex.\\
    \emph{Inductive Hypothesis: }Assume $\forall l < n, v = 2n - 1$\\
    \emph{Inductive Step: } l = n, let us remove the root r, the tree T will break into two trees L and R. 
    Suppose L has $n_1$ leaves and $v_1$ vertices, R has $n_2$ leaves and $v_2$ vertices. 
    $v = v_1 + v_2 = 2*n_1 - 1 + 2*n_2 - 1 + 1 = 2*n - 1$

\part*{Euclid's Algorithm}
\begin{enumerate}[(a).]
    \item Calculation table lists below.\\
    \begin{tabular}{|c|c|c|}
    \hline
    Function Calls&(x,y)&x mod y\\
    \hline
    1&(527,323)&204\\
    \hline
    2&(323,204)&119\\
    \hline
    3&(204,119)&85\\
    \hline
    4&(119,85)&34\\
    \hline
    5&(85,34)&17\\
    \hline
    6&(34,17)&0\\
    \hline
    7&(17,0)&--\\
    \hline
    \end{tabular}
    \\
    Hence the gcd(527,323) = 17.

    \item Compute the multiplicative inverse of 5 mod 27.\\
    \begin{tabular}{|c|c|c|c|}
        \hline
        Function Calls&(x,y)&x mod y&x div y\\
        \hline
        1&(27,5)&2&5\\
        \hline
        2&(5,2)&1&2\\
        \hline
        3&(2,1)&0&2\\
        \hline
        4&(1,0)&--&--\\
        \hline
    \end{tabular}
    \\
    Then the returned values of all recursive calls are:\\
    \begin{tabular}{|c|c|c|}
        \hline
        Function Calls&(d,a,b)&Returned Values\\
        \hline
        1&--&(1,1,0)\\
        \hline
        2&(1,1,0)&(1,0,1)\\
        \hline
        3&(1,0,1)&(1,1,-2)\\
        \hline
        4&(1,1,-2)&(1,-2,11)\\
        \hline
    \end{tabular}
    \\
    Thus the the multiplicative inverse of 5 mod 27 is 11.

    \item Find x (mod 27).\\
    \begin{align*}
        5x + 26 \equiv 3 \mod{27} &\implies 5x \equiv -23 \mod{27}\\
        &\implies 5x \equiv 4 \mod{27}\\
        &\implies 11 \times 5x \equiv 11 \times 4 \mod{27}\\
        &\implies x \equiv 44 \mod{27}\\
        &\implies x \equiv 17 \mod{27}
    \end{align*}

    \item A counterexample is a = 3, b = 6, c = 12. While a has no multiplicative inverse mod c, 
    but we can get a x = 2 for $ax \equiv b \mod{12}$.
\end{enumerate}

\part*{GCD Proof}

\emph{Solution:} \\
1. Prove forward.\\
Assume there is a gcd d that is greater than 1.\\
\begin{align*}
    ax &\equiv 1 \mod{n}\\
    ax &= bn + 1\\
    \frac{ax}{d} &= \frac{bn}{d} + \frac{1}{d}\\
\end{align*}
ax/d and bn/d must both be integers, so 1/d is also an integer, but d > 1, thus 1/d is not an integer, contradict. 
\\

2. Prove backward.\\
We run egcd on n and x, the output is $a,b \in \mathbb{Z}$ and $d = gcd(n,x) = 1$, such that 
\begin{align*}
    an + bx &= 1\\
    bx &\equiv 1 \mod{n}
\end{align*}
\\Thus, x has a multiplicative inverse \emph{b}.





\end{enumerate}





\end{document}