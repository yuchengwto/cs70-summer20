\documentclass{article}
\usepackage{mathtools}
\usepackage{amsfonts}
\usepackage{amssymb}
\usepackage{enumerate}
\usepackage{amsthm}
\usepackage{amsmath}
\usepackage{listings}
\newcommand{\powerset}[1]{\mathbb{P}(#1)}

\begin{document}
\part*{Modular Exponentiation}
    \begin{enumerate}[(a).]
        \item 8
        \item 160 = 32 + 128, 16
        \item $218^3 = (200 + 10 + 8)^3 = 8 \mod{9}$
        \item $998^156 = (998^12)^13 = 1 \mod{13}$
    \end{enumerate}    

\part*{Sparsity of Primes}
Provement equals to find a x that x+1, x+2, ... x+k are all not powers of primes. 
If a number can be divided by two distinct prime, then this number is not a prime power.  
This property can also apply to x+1, ... x+k these k integers. So we can select 2k distinct primes 
p1, p2, ... p2k-1, p2k and enforce the following constraints:\\

\begin{align*}
    x + 1 &\equiv 0 \mod{p_1p_2}\\
    x + 2 &\equiv 0 \mod{p_3p_4}\\
    \cdots\\
    x + k &\equiv 0 \mod{p_{2k-1}p_{2k}}
\end{align*}

By Chinese Reminder Theorem, we can calculate the value of x 
so this x must exists, and thus x + 1 through x + k are all not prime powers. 


\part*{LOTUS but for CRT}
Since p and q are primes, so gcd(a, p) = 1 and gcd(a, q) = 1. 
By FLT,
\[a^{(p-1)(q-1)+1} = (a^{p-1})^{q-1} \cdot a \equiv 1^{q-1} \cdot a \equiv a \mod{p} \]
\[a^{(p-1)(q-1)+1} = (a^{q-1})^{p-1} \cdot a \equiv 1^{p-1} \cdot a \equiv a \mod{q} \]

Consider the system of congruences,
\[x \equiv a \mod{p}\]
\[x \equiv a \mod{q}\]

By CRT, we can calculate a value of x,
\[x = a\cdot q^{-1}(\mod{p}) \cdot q + a\cdot p^{-1}(\mod{q})\cdot p (\mod{pq})\]
\[x = a(\cdot q^{-1}(\mod{p}) \cdot q + p^{-1}(\mod{q})\cdot p) (\mod{pq})\]

Since p and q are co-prime, by Bezout’s lemma, 
\[g\cdot p + h\cdot q = 1\]
where g and h are respectively $p^{-1}(\mod{q})$ and $q^{-1}(\mod{p})$. Hence, 
\[x = a(g\cdot p + h\cdot q) \equiv a\cdot 1 \mod{pq}\]

Therefore, let $x = a^{(p-1)(q-1)+1}$, $x \equiv a \mod{p}$ and 
$x \equiv a \mod{q}$, then $x \equiv a \mod{pq}$. We conclude that $a^{(p-1)(q-1)+1} \equiv a \mod{pq}$. 

\part*{Squared RSA}
\begin{enumerate}[(a).]
    \item Prove the identity $a^{p(p-1)} \equiv 1 \mod{p^2}$ where 
    a is coprime to p and p is a prime.\\
    Let \[S = \{1, 2, 3, \cdots p^{2}-1\}\] where each element is coprime to p. 
    
    and \[T = \{a, 2a, 3a, \cdots a\cdot p^{2}-1\}\]

    Exclude $\{p^2-p, p^2-2p, \cdots p^2-(p-1)p\}$ from 1 to $p^{2}-1$, 
    we know that $|S| = p^2 - 1 - (p-1) = p(p-1)$.\\

    Let prove $S = T$:
    \begin{itemize}
        \item $S \subseteq T$: Since gcd(a, p) = 1, 
        $a^{-1} \mod{p^2}$ exists, and also gcd($a^{-1}$, p) = 1. 
        Let $x \in S$, and gcd(x, p) = 1, so gcd($a^{-1}x$, p) = 1, so $a^{-1}x \in S$. 
        $a(a^{-1}x) = x \in T$. Hence $S \subseteq T$.
        \item $T \subseteq T$: Let $ax \in T$ where $x \in S$, we know that 
        gcd(x, p) = 1, gcd(a, p) = 1, and ax is also coprime to p as well. 
        Since S include all distinct numbers from 1 to $p^{2}-1$ where each number is coprime to p. 
        So $ax \in S$. We conclude.
    \end{itemize}

    Since S = T, we know that, 
    \[ \prod\limits_{s_i \in S} s_i = \prod\limits_{t_i \in T} t_i \mod{p^2} \]
    \[ \prod\limits_{t_i \in T} t_i = \prod\limits_{s_i \in S} s_i\cdot a = a^{|S|}\cdot \prod\limits_{s_i \in S} s_i = \prod\limits_{s_i \in S} s_i \mod{p^2} \]

    Since each $s_i \in S$ is coprime to p, $\prod\limits_{s_i \in S} s_i$ is also coprime to $p^{2}$. 
    Hence we can multiply both sides of our equivalence with the inverse of $\prod\limits_{s_i \in S} s_i \mod{p^2}$ 
    to obtain $a^{p(p-1)} \equiv 1 \mod{p^2}$. Thus we conclude.

    \item Prove $(x^e)^d \equiv x \mod{p^2q^2}$.\\
    We know that $ed \equiv 1 \mod{p(p-1)q(q-1)}$, thus 
    $ed = 1 + kpq(p-1)(q-1)$. Our claim is that $x^{ed} - x \equiv 1 \mod{pq(p-1)(q-1)}$. \\
    
    $x^{ed} - x = x(x^{kpq(p-1)(q-1)} - 1)$\\

    Now we claim that the expression $x(x^{kpq(p-1)(q-1)} - 1)$ is divisible by $p^2$. 
    To see this, we consider two case: 
    \begin{itemize}
        \item \emph{Case 1: }x is divisible by $p^2$, the expression is clearly divisible by $p^2$. 
        \item \emph{Case 2: }x is not divisible by $p^2$, by FLT and part a, the expression $\equiv x(1^{kq(q-1)} - 1) \equiv 0 \mod{p^2}$.
    \end{itemize}
    By an entirely symmetrical argument, the expression is also divisible by $q^2$. 
    Since p and q is primes, the expression must be divisible by $p^2q^2$. Thus, we conclude.

\end{enumerate}


\part*{Polynomials over Galois Fields}
\begin{enumerate}[(a).]
    \item \begin{align*}
        q(x) &= x^p - x \mod{p}\\
        q(x) &= x(x^{p-1} - x) \mod{p}\\
        q(x) &= 0 \mod{p}\\
        q(x) &= \prod\limits_{k=0}^{p-1} (x-k)\\
    \end{align*}
    
    \item By Lagrange interpolation, passing through p points (0, f(0)), (1, f(0)), ... (p-1, f(p-1)) 
    there is a unique polynomial of at most p-1 degree $\widetilde{f}(x)$.\\
    
    Alternatively, by FLT, let d >= p, we know that, 
    \begin{align*}
        x^d &= x^{d-(p-1)+(p-1)}\\
        &\equiv x^{d-(p-1)} \mod{p}\\
        &\equiv x^{d-2(p-1)} \mod{p}\\
        &\cdots
    \end{align*}
    So there must be a integer k to let $d-k(p-1) < p$, and $x^d \equiv x^{d-k(p-1)} \mod{p}$ by this k. 
    We can apply this property to each $x^n$ in f(x) where n >= p and obtain a polynomial with at most p-1 degree. 


    \item Lemma: \emph{The roots of R(x)=P(x)Q(x) are the union of the roots of P and Q}, the above claim clearly holds. \\
    Let U be the union of the roots of P and Q, $\forall x \in \mathbb{R}$, there are two cases below: 
    \begin{itemize}
        \item \emph{$x \in U$: }Hence $P(x) = 0 \vee Q(x) = 0$, then $R(x) = 0$. 
        \item \emph{$x \notin U$: }Hence $P(x) \not= 0 \wedge Q(x) \not=0$, then $R(x) \not= 0$. 
    \end{itemize}
    Therefore, the lemma holds.

    Suppose that P(x) and Q(x) are both non-zero polynomials and only have limited number of roots, hence R(x) also has limited roots. 
    By contrapositive, since $x \in \mathbb{R}$, there must exists a $x_{nozero}$ with which $P(x_{nozero}) \not= 0$ and $Q(x_{nozero})\not=0$, then $P(x_{nozero})Q(x_{nozero})\not=0$. 

    \item In GF(p), $x^{p-1} - 1$ and x are both non zero polynomials, but their product $x^p - x \equiv 0 \mod{p}$ by FLT. 
\end{enumerate}

\part*{Packet Requirements}
\begin{enumerate}[(a).]
    \item Suppose Bob get n+2k-1 packets, where exists k corrupted packets.\\

    If Bob select n uncorrupted packets, and get the interpolated polynomial f. 
    Then f will pass through k-1 packets additionally.\\

    If Bob select n packets consist of c corrupted packets and n-c uncorrupted packets. 
    The remaining 2k-1 packets contains k-1+c uncorrupted packets and k-c corrupted. 
    Note that the interpolated polynomial can pass through at most n-1 corrupted packets. 
    So the polynomial can additionally pass through (n-1)-(n-c)=c-1 uncorrupted packets, and k-c corrupted packets, 
    that is additional c-1 + k-c = k-1 packets, which is the same as the correct case above. 
    Hence Bob cannot distinguish.

    \item Suppose Bob get n+2k packets, where exists k corrupted packets.\\
    
    If Bob select n uncorrupted packets and get the corrected polynomial. 
    Then polynomial will pass through k packets additionally.\\

    If Bob select n packets consist of c corrupted packets and n-c uncorrupted packets. 
    The remaining 2k packets contains k+c uncorrupted packets and k-c corrupted. 
    Note that the interpolated polynomial can pass through at most n-1 corrupted packets. 
    So the polynomial can additionally pass through (n-1)-(n-c)=c-1 uncorrupted packets, and k-c corrupted packets, 
    that is additional c-1 + k-c = k-1 packets, which is different from k in the correct case. 
    Hence Bob can distinguish.

\end{enumerate}

\part*{ALice and Bob}

\begin{enumerate}[(a).]
    \item He can recover the origin message. $Q(x) = x^3 +5x^2+5x+4,E(x) = x−3$. 
    Hence the $P(x) = x^2 + x + 1$, the x-value of the packet Eve changed is 3. 
    
    \item Since Bob know that there still remains 3 points uncorrupted and 2 points are corrupted. 
    So the remaining 3 corrected points are always on the degree 1 polynomial that Alice encoded her message on. 
    If Bob find multiple degree 1 polynomial across 3 points, then he will not be able to determine Alice's message, 
    but if only one is found, he can. \\

    In this case, if $x = 5, 6, 10$, Bob cannot do well. 

    \item Alice can compile message 1 to 6 to a degree 5 polynomial and send 10 points about the polynomial to channel A. 
    Then compile message 7 to 9 to a degree 2 polynomial and send 10 points about the polynomial to channel B. 
    Bob can decode the degree 5 polynomial from 6 points from channel A and 
    the degree 2 polynomial from 5 uncorrupted points and 1 corrupted point from channel B. 
    For channel b, it can effectively pass a degree 3 polynomial with 5 uncorrupted points and one corrupted. 
    So, there is even some redundancy. 
\end{enumerate}


\end{document}