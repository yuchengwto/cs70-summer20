\documentclass{article}
\usepackage{mathtools}
\usepackage{amsfonts}
\usepackage{amssymb}
\usepackage{enumerate}
\usepackage{amsthm}
\usepackage{amsmath}
\newcommand{\powerset}[1]{\mathbb{P}(#1)}

\begin{document}
\part*{Countability Practice}
    \begin{enumerate}[(a).]
        \item Yes, they have the same cardinality.
        \[f(x) = \frac{1}{x} - 1\]
        Suppose f(x) = f(y)\\
        \begin{align*}
            &\frac{1}{x} - 1 = \frac{1}{y} - 1\\
            &x = y\\
        \end{align*}
        Hence f is one to one.\\
        \\
        Take any $y \in (0, \infty)$, let $x \in (0, 1), x = 1/(1+y)$. Then,
        \[f(x) = \frac{1}{x} - 1 \]
        \[f(x) = \frac{1}{\frac{1}{1+y}} - 1\]
        \[f(x) =  y\]
        So f maps x to y, hence onto.

        \item Yes, it is countable. We can enumerate the strings in 
        such a way that each string appears exactly once in the list. 
        Firstly, list all strings of length 1 in lexicographic order, then 
        all strings of length 2, and then length 3, and so on. Since each step, 
        there are only finite number of strings of a particular length, any string 
        of finite length appears in the list, and it is clear that each string 
        appears exactly once in this list.

        
        


    \end{enumerate}    



\end{document}